\documentclass[aspectratio=169]{beamer}
\title{McElreath \& Boyd (2007)}
\subtitle{Chapter 1: The Theoretician's Laboratory}
\author{Bram Kuijper}
\institute{University of Exeter, Penryn Campus}
\date{\today}
\begin{document}

\frame{\titlepage}


\begin{frame}
\frametitle{Purpose of the book / these lectures}
\begin{itemize}
    \item Increase your understanding about the purpose of theory in evolutionary biology 
    \item Getting to grips with key mathematical techniques
    \item Understanding how to think conceptually about major questions in social evolution
    \item Learn how to build models yourself
\end{itemize}
\end{frame}


\begin{frame}
\frametitle{Prior knowledge}
\begin{itemize}
    \item Their words (MB2007, page 2)
    \item Getting to grips with key mathematical techniques
    \item Understanding how to think conceptually about major questions in social evolution
    \item Learn how to build models yourself
\end{itemize}
\end{frame}


\begin{frame}
\frametitle{How to study this book?}
    \emph{See comments on pages x-xi in McElreath \& Boyd}
\begin{itemize}
    \item Read the book in order (skipping parts will be difficult)
    \item Use pen and paper whilst reading the book: make sure you can derive equations yourself
    \item Work through all/most of the problems (yes this takes time!)
    \item Be patient: if you do not understand something, do not conclude this is too hard for you. Rather, ask for help, and try to rederive the relevant material multiple times
\end{itemize}
\end{frame}
\end{document}
