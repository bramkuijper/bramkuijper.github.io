% Options for packages loaded elsewhere
\PassOptionsToPackage{unicode}{hyperref}
\PassOptionsToPackage{hyphens}{url}
%
\documentclass[
]{book}
\usepackage{amsmath,amssymb}
\usepackage{iftex}
\ifPDFTeX
  \usepackage[T1]{fontenc}
  \usepackage[utf8]{inputenc}
  \usepackage{textcomp} % provide euro and other symbols
\else % if luatex or xetex
  \usepackage{unicode-math} % this also loads fontspec
  \defaultfontfeatures{Scale=MatchLowercase}
  \defaultfontfeatures[\rmfamily]{Ligatures=TeX,Scale=1}
\fi
\usepackage{lmodern}
\ifPDFTeX\else
  % xetex/luatex font selection
\fi
% Use upquote if available, for straight quotes in verbatim environments
\IfFileExists{upquote.sty}{\usepackage{upquote}}{}
\IfFileExists{microtype.sty}{% use microtype if available
  \usepackage[]{microtype}
  \UseMicrotypeSet[protrusion]{basicmath} % disable protrusion for tt fonts
}{}
\makeatletter
\@ifundefined{KOMAClassName}{% if non-KOMA class
  \IfFileExists{parskip.sty}{%
    \usepackage{parskip}
  }{% else
    \setlength{\parindent}{0pt}
    \setlength{\parskip}{6pt plus 2pt minus 1pt}}
}{% if KOMA class
  \KOMAoptions{parskip=half}}
\makeatother
\usepackage{xcolor}
\usepackage{color}
\usepackage{fancyvrb}
\newcommand{\VerbBar}{|}
\newcommand{\VERB}{\Verb[commandchars=\\\{\}]}
\DefineVerbatimEnvironment{Highlighting}{Verbatim}{commandchars=\\\{\}}
% Add ',fontsize=\small' for more characters per line
\usepackage{framed}
\definecolor{shadecolor}{RGB}{248,248,248}
\newenvironment{Shaded}{\begin{snugshade}}{\end{snugshade}}
\newcommand{\AlertTok}[1]{\textcolor[rgb]{0.94,0.16,0.16}{#1}}
\newcommand{\AnnotationTok}[1]{\textcolor[rgb]{0.56,0.35,0.01}{\textbf{\textit{#1}}}}
\newcommand{\AttributeTok}[1]{\textcolor[rgb]{0.13,0.29,0.53}{#1}}
\newcommand{\BaseNTok}[1]{\textcolor[rgb]{0.00,0.00,0.81}{#1}}
\newcommand{\BuiltInTok}[1]{#1}
\newcommand{\CharTok}[1]{\textcolor[rgb]{0.31,0.60,0.02}{#1}}
\newcommand{\CommentTok}[1]{\textcolor[rgb]{0.56,0.35,0.01}{\textit{#1}}}
\newcommand{\CommentVarTok}[1]{\textcolor[rgb]{0.56,0.35,0.01}{\textbf{\textit{#1}}}}
\newcommand{\ConstantTok}[1]{\textcolor[rgb]{0.56,0.35,0.01}{#1}}
\newcommand{\ControlFlowTok}[1]{\textcolor[rgb]{0.13,0.29,0.53}{\textbf{#1}}}
\newcommand{\DataTypeTok}[1]{\textcolor[rgb]{0.13,0.29,0.53}{#1}}
\newcommand{\DecValTok}[1]{\textcolor[rgb]{0.00,0.00,0.81}{#1}}
\newcommand{\DocumentationTok}[1]{\textcolor[rgb]{0.56,0.35,0.01}{\textbf{\textit{#1}}}}
\newcommand{\ErrorTok}[1]{\textcolor[rgb]{0.64,0.00,0.00}{\textbf{#1}}}
\newcommand{\ExtensionTok}[1]{#1}
\newcommand{\FloatTok}[1]{\textcolor[rgb]{0.00,0.00,0.81}{#1}}
\newcommand{\FunctionTok}[1]{\textcolor[rgb]{0.13,0.29,0.53}{\textbf{#1}}}
\newcommand{\ImportTok}[1]{#1}
\newcommand{\InformationTok}[1]{\textcolor[rgb]{0.56,0.35,0.01}{\textbf{\textit{#1}}}}
\newcommand{\KeywordTok}[1]{\textcolor[rgb]{0.13,0.29,0.53}{\textbf{#1}}}
\newcommand{\NormalTok}[1]{#1}
\newcommand{\OperatorTok}[1]{\textcolor[rgb]{0.81,0.36,0.00}{\textbf{#1}}}
\newcommand{\OtherTok}[1]{\textcolor[rgb]{0.56,0.35,0.01}{#1}}
\newcommand{\PreprocessorTok}[1]{\textcolor[rgb]{0.56,0.35,0.01}{\textit{#1}}}
\newcommand{\RegionMarkerTok}[1]{#1}
\newcommand{\SpecialCharTok}[1]{\textcolor[rgb]{0.81,0.36,0.00}{\textbf{#1}}}
\newcommand{\SpecialStringTok}[1]{\textcolor[rgb]{0.31,0.60,0.02}{#1}}
\newcommand{\StringTok}[1]{\textcolor[rgb]{0.31,0.60,0.02}{#1}}
\newcommand{\VariableTok}[1]{\textcolor[rgb]{0.00,0.00,0.00}{#1}}
\newcommand{\VerbatimStringTok}[1]{\textcolor[rgb]{0.31,0.60,0.02}{#1}}
\newcommand{\WarningTok}[1]{\textcolor[rgb]{0.56,0.35,0.01}{\textbf{\textit{#1}}}}
\usepackage{longtable,booktabs,array}
\usepackage{calc} % for calculating minipage widths
% Correct order of tables after \paragraph or \subparagraph
\usepackage{etoolbox}
\makeatletter
\patchcmd\longtable{\par}{\if@noskipsec\mbox{}\fi\par}{}{}
\makeatother
% Allow footnotes in longtable head/foot
\IfFileExists{footnotehyper.sty}{\usepackage{footnotehyper}}{\usepackage{footnote}}
\makesavenoteenv{longtable}
\usepackage{graphicx}
\makeatletter
\def\maxwidth{\ifdim\Gin@nat@width>\linewidth\linewidth\else\Gin@nat@width\fi}
\def\maxheight{\ifdim\Gin@nat@height>\textheight\textheight\else\Gin@nat@height\fi}
\makeatother
% Scale images if necessary, so that they will not overflow the page
% margins by default, and it is still possible to overwrite the defaults
% using explicit options in \includegraphics[width, height, ...]{}
\setkeys{Gin}{width=\maxwidth,height=\maxheight,keepaspectratio}
% Set default figure placement to htbp
\makeatletter
\def\fps@figure{htbp}
\makeatother
\setlength{\emergencystretch}{3em} % prevent overfull lines
\providecommand{\tightlist}{%
  \setlength{\itemsep}{0pt}\setlength{\parskip}{0pt}}
\setcounter{secnumdepth}{5}
\usepackage{booktabs}
\ifLuaTeX
  \usepackage{selnolig}  % disable illegal ligatures
\fi
\usepackage[]{natbib}
\bibliographystyle{plainnat}
\usepackage{bookmark}
\IfFileExists{xurl.sty}{\usepackage{xurl}}{} % add URL line breaks if available
\urlstyle{same}
\hypersetup{
  pdftitle={Epidemiological Modelling: Course Notes},
  pdfauthor={Bram Kuijper, Mario Recker},
  hidelinks,
  pdfcreator={LaTeX via pandoc}}

\title{Epidemiological Modelling: Course Notes}
\author{Bram Kuijper, Mario Recker}
\date{2024-02-27}

\usepackage{amsthm}
\newtheorem{theorem}{Theorem}[chapter]
\newtheorem{lemma}{Lemma}[chapter]
\newtheorem{corollary}{Corollary}[chapter]
\newtheorem{proposition}{Proposition}[chapter]
\newtheorem{conjecture}{Conjecture}[chapter]
\theoremstyle{definition}
\newtheorem{definition}{Definition}[chapter]
\theoremstyle{definition}
\newtheorem{example}{Example}[chapter]
\theoremstyle{definition}
\newtheorem{exercise}{Exercise}[chapter]
\theoremstyle{definition}
\newtheorem{hypothesis}{Hypothesis}[chapter]
\theoremstyle{remark}
\newtheorem*{remark}{Remark}
\newtheorem*{solution}{Solution}
\begin{document}
\maketitle

{
\setcounter{tocdepth}{1}
\tableofcontents
}
\chapter{Lecture Notes from uncertainties to curve fitting}\label{lecture-notes-from-uncertainties-to-curve-fitting}

Here the lecture notes with hints to use during the tutorials and the questions for you to solve during the practical.

\section{}\label{section}

\section{Copying code}\label{copying-code}

If you hover over the top right corner of a code block in these lecture notes, a copy button emerges that allows
you to copy the code (and then paste it into Rstudio) without having to first select all of the code.

\chapter{Tutorial uncertainties}\label{tutorial-uncertainties}

\section{\texorpdfstring{Estimating \(R_{0}\) in a risk-heterogenous population}{Estimating R\_\{0\} in a risk-heterogenous population}}\label{estimating-r_0-in-a-risk-heterogenous-population}

In the first of this tutorial, we will run an SIS model
of a population heterogenous for risk. We then estimate \(R_{0}\) for different WAIFW matrices (Who Acquires Infection From Whom)

\subsection{SIS model code}\label{sis-model-code}

Inspect the model code below (particularly the set of
differential equations) and compare the equations with
the flow diagram of the SIS model in the lecture slides.

\begin{Shaded}
\begin{Highlighting}[]
\FunctionTok{library}\NormalTok{(}\StringTok{"tidyverse"}\NormalTok{)}
\FunctionTok{library}\NormalTok{(}\StringTok{"deSolve"}\NormalTok{)}

\CommentTok{\# set of differential equations }
\CommentTok{\# of a model in which classes vary}
\CommentTok{\# in their transmission risk}
\NormalTok{SIS\_structure }\OtherTok{\textless{}{-}} \ControlFlowTok{function}\NormalTok{(t, y, parameters)}
\NormalTok{\{}
    \CommentTok{\# get state variables from}
    \CommentTok{\# the y function argument}
\NormalTok{    SL }\OtherTok{=}\NormalTok{ y[}\DecValTok{1}\NormalTok{]}
\NormalTok{    SH }\OtherTok{=}\NormalTok{ y[}\DecValTok{2}\NormalTok{]}
\NormalTok{    IL }\OtherTok{=}\NormalTok{ y[}\DecValTok{3}\NormalTok{]}
\NormalTok{    IH }\OtherTok{=}\NormalTok{ y[}\DecValTok{4}\NormalTok{]}
    
    \CommentTok{\# with() creates a \textquotesingle{}sub environment\textquotesingle{}}
    \CommentTok{\# in which all the named values of the }
    \CommentTok{\# parameters list are now variables by themselves}
    \CommentTok{\# hence, we can then evaluate all the gradients}
    \CommentTok{\# within this environment}
\NormalTok{    result }\OtherTok{\textless{}{-}} \FunctionTok{with}\NormalTok{(}\AttributeTok{data=}\FunctionTok{as.list}\NormalTok{(parameters),}
    \AttributeTok{expr =}\NormalTok{ \{}
        \CommentTok{\# evaluate each of the equations}
\NormalTok{        dSL }\OtherTok{\textless{}{-}} \SpecialCharTok{{-}}\NormalTok{mu }\SpecialCharTok{*}\NormalTok{ SL }\SpecialCharTok{{-}}\NormalTok{ betaLL }\SpecialCharTok{*}\NormalTok{ SL }\SpecialCharTok{*}\NormalTok{ IL }\SpecialCharTok{{-}}\NormalTok{ betaHL }\SpecialCharTok{*}\NormalTok{ SL }\SpecialCharTok{*}\NormalTok{ IH }\SpecialCharTok{+}\NormalTok{ gamma }\SpecialCharTok{*}\NormalTok{ IL}
        
\NormalTok{        dSH }\OtherTok{\textless{}{-}} \SpecialCharTok{{-}}\NormalTok{mu }\SpecialCharTok{*}\NormalTok{ SH }\SpecialCharTok{{-}}\NormalTok{ betaLH }\SpecialCharTok{*}\NormalTok{ SH }\SpecialCharTok{*}\NormalTok{ IL }\SpecialCharTok{{-}}\NormalTok{ betaHH }\SpecialCharTok{*}\NormalTok{ SH }\SpecialCharTok{*}\NormalTok{ IH }\SpecialCharTok{+}\NormalTok{ gamma }\SpecialCharTok{*}\NormalTok{ IH}
        
\NormalTok{        dIL }\OtherTok{\textless{}{-}}\NormalTok{ betaHL }\SpecialCharTok{*}\NormalTok{ SL }\SpecialCharTok{*}\NormalTok{ IH }\SpecialCharTok{+}\NormalTok{ betaLL }\SpecialCharTok{*}\NormalTok{ SL }\SpecialCharTok{*}\NormalTok{ IL }\SpecialCharTok{{-}}\NormalTok{ (gamma  }\SpecialCharTok{+}\NormalTok{ mu) }\SpecialCharTok{*}\NormalTok{ IL}
        
\NormalTok{        dIH }\OtherTok{\textless{}{-}}\NormalTok{ betaLH }\SpecialCharTok{*}\NormalTok{ SH }\SpecialCharTok{*}\NormalTok{ IL }\SpecialCharTok{+}\NormalTok{ betaHH }\SpecialCharTok{*}\NormalTok{ SH }\SpecialCharTok{*}\NormalTok{ IH }\SpecialCharTok{{-}}\NormalTok{ (gamma  }\SpecialCharTok{+}\NormalTok{ mu) }\SpecialCharTok{*}\NormalTok{ IH}
       
        \CommentTok{\# then return the gradient values}
        \FunctionTok{c}\NormalTok{(dSL,dSH,dIL,dIH)}
\NormalTok{    \})}
    
    \CommentTok{\# return list of gradients }
    \FunctionTok{return}\NormalTok{(}\FunctionTok{list}\NormalTok{(result))}
\NormalTok{\} }\CommentTok{\# end SIR\_structure}

\CommentTok{\# set out the time points }
\NormalTok{times }\OtherTok{\textless{}{-}} \FunctionTok{seq}\NormalTok{(}\AttributeTok{from =} \DecValTok{0}\NormalTok{, }\AttributeTok{to =} \DecValTok{25}\NormalTok{, }\AttributeTok{length.out =} \DecValTok{1000}\NormalTok{)}

\CommentTok{\# make a named vector containing all }
\CommentTok{\# the parameters used in the model}
\NormalTok{params }\OtherTok{\textless{}{-}} \FunctionTok{c}\NormalTok{(}
    \AttributeTok{betaHH=}\DecValTok{5}\NormalTok{,}
    \AttributeTok{betaHL=}\FloatTok{0.1}\NormalTok{,}
    \AttributeTok{betaLH=}\FloatTok{0.1}\NormalTok{,}
    \AttributeTok{betaLL=}\DecValTok{2}\NormalTok{,}
    \AttributeTok{gamma=}\DecValTok{1}\NormalTok{,}
    \AttributeTok{mu=}\DecValTok{0}\NormalTok{,}
    \AttributeTok{N=}\DecValTok{1}
\NormalTok{)}

\CommentTok{\# a set of starting values in which we}
\CommentTok{\# assume that equally frequent}
\NormalTok{start }\OtherTok{\textless{}{-}} \FunctionTok{c}\NormalTok{(}\AttributeTok{SL=}\FloatTok{0.495}\NormalTok{,}\AttributeTok{SH=}\FloatTok{0.495}\NormalTok{,}\AttributeTok{IL=}\FloatTok{0.005}\NormalTok{,}\AttributeTok{IH=}\FloatTok{0.005}\NormalTok{)}

\CommentTok{\# run the model and obtain a data.frame}
\CommentTok{\# with densities over time}
\NormalTok{output }\OtherTok{\textless{}{-}} \FunctionTok{as.data.frame}\NormalTok{(}\FunctionTok{ode}\NormalTok{(}\AttributeTok{y =}\NormalTok{ start}
\NormalTok{              ,}\AttributeTok{times =}\NormalTok{ times}
\NormalTok{              ,}\AttributeTok{func=}\NormalTok{SIS\_structure}
\NormalTok{              ,}\AttributeTok{parms =}\NormalTok{ params))}
\end{Highlighting}
\end{Shaded}

\subsection{Task: estimate R0 without taking into account risk}\label{task-estimate-r0-without-taking-into-account-risk}

After having inspected the model code, try to run it.
There should now be an \texttt{output} variable, which contains a \texttt{data.frame}. This \texttt{data.frame} contains the
the time evolution of the densities of susceptibles
and infecteds.

Try to get the total proportion of susceptibles from the \texttt{output} \texttt{data.frame}, by summing the low and high
risk susceptibles. Use this number as your value of
\(p_{S}\) and then calculate \(R_{0}\) from that (see
the lecture slides for the corresponding formula).
\emph{Hint}: you need to use the equilibrium values of the
number of susceptibles.

\subsection{Task: estimate R0 while taking into account risk}\label{task-estimate-r0-while-taking-into-account-risk}

Now we will try to obtain a more precise measure of
\(R_{0}\), by using the WAIFW matrix and the
equilibrium values of \(S_{H}\) and \(S_{L}\) which you obtained in the previous subsection.

Use R's \texttt{matrix()} command to fill out the following
\(2 \times 2\) matrix, using the densities \(S_{H}\), \(S_{L}\) and the parameter values of \(\mathbf{beta}\) that you used
to run the SIS model

\[\begin{aligned}
\mathbf{R} &= \left [ \begin{matrix} \beta_{HH} S_{H}, \beta_{HL} S_{L} \\
\beta_{LH} S_{L}, \beta_{LL} S_{L}
\end{matrix} \right ]
\end{aligned}\]

Then use the \texttt{eigen()} command on this matrix to calculate
the dominant (i.e., the largest) eigenvalue, which is \(R_{0}\) as it calculates the overall number of secondary
cases, while taking into account the different contributions of high and low risk individuals.

Associated to the dominant eigenvalue is the dominant eigenvector, absolute
values of which inform one about the relative long-term contribution in spreading the epidemic by high versus low-risk individuals.

\section{Maximum Likelihood and age-dependent seropositivity}\label{maximum-likelihood-and-age-dependent-seropositivity}

For a disease with an unknown \(R_{0}\), we provide
three datasets with ages and whether individuals are seropositive
or seronegative. We will then use a maximum likelihood
approach to estimate the value of \(R_{0}\) that results from this data.

\subsection{Probabilities}\label{probabilities}

Let \(P(a_{i}) = \exp \left [-a_{i}\mu \left (R_{0} - 1 \right ) \right ]\) be the probability that the \(i\)th individual sampled of age \(a_{i}\) is still susceptible (i.e., has no antibodies). Similarly, \(Q(a_{i}) = 1 - \exp \left [-a_{i}\mu \left (R_{0} - 1 \right ) \right ]\) reflects the probability

\chapter{Parts}\label{parts}

You can add parts to organize one or more book chapters together. Parts can be inserted at the top of an .Rmd file, before the first-level chapter heading in that same file.

Add a numbered part: \texttt{\#\ (PART)\ Act\ one\ \{-\}} (followed by \texttt{\#\ A\ chapter})

Add an unnumbered part: \texttt{\#\ (PART\textbackslash{}*)\ Act\ one\ \{-\}} (followed by \texttt{\#\ A\ chapter})

Add an appendix as a special kind of un-numbered part: \texttt{\#\ (APPENDIX)\ Other\ stuff\ \{-\}} (followed by \texttt{\#\ A\ chapter}). Chapters in an appendix are prepended with letters instead of numbers.

\chapter{Fitting dynamic models to data}\label{fitting-dynamic-models-to-data}

\section{Hong Kong flu dataset}\label{hong-kong-flu-dataset}

Please find a dataset on the 1968 Hong Kong flu outbreak in New York below:

\begin{Shaded}
\begin{Highlighting}[]
\NormalTok{flu }\OtherTok{\textless{}{-}} \FunctionTok{data.frame}\NormalTok{(}\AttributeTok{week =} \DecValTok{1}\SpecialCharTok{:}\DecValTok{13}\NormalTok{,}
                  \AttributeTok{deaths =} \FunctionTok{c}\NormalTok{(}\DecValTok{14}\NormalTok{, }\DecValTok{28}\NormalTok{, }\DecValTok{50}\NormalTok{, }\DecValTok{66}\NormalTok{, }\DecValTok{156}\NormalTok{, }\DecValTok{190}\NormalTok{, }\DecValTok{156}\NormalTok{, }\DecValTok{108}\NormalTok{, }\DecValTok{68}\NormalTok{, }\DecValTok{77}\NormalTok{, }\DecValTok{33}\NormalTok{, }\DecValTok{65}\NormalTok{, }\DecValTok{24}\NormalTok{))}
\end{Highlighting}
\end{Shaded}

We will now use \texttt{optim()} to try and fit this dataset
and estimate values for the transmission rate \(\beta\)
and the disease clearance rate \(\gamma\).

\subsection{Setting up the SIR model}\label{setting-up-the-sir-model}

First, we code up a SIR ODE model with frequency-dependent transmission (as in the previous practicals) that can
be solved by \texttt{deSolve}'s \texttt{ode()} function.
Again, we ignore demography similar to the SIR model presented on day 1.
For example, this would be the skeleton of such
a function:

\begin{Shaded}
\begin{Highlighting}[]
\NormalTok{sir\_ode }\OtherTok{\textless{}{-}} \ControlFlowTok{function}\NormalTok{(t, demographic\_variables, parameters)}
\NormalTok{\{}
    \FunctionTok{with}\NormalTok{(}\FunctionTok{as.list}\NormalTok{(demographic\_variables, parameters)}
\NormalTok{         \{}
\NormalTok{           dS }\OtherTok{\textless{}{-}} \SpecialCharTok{{-}}\NormalTok{beta }\SpecialCharTok{*}\NormalTok{ S }\SpecialCharTok{*}\NormalTok{ I }\SpecialCharTok{/}\NormalTok{(S }\SpecialCharTok{+}\NormalTok{ I }\SpecialCharTok{+}\NormalTok{ R) }
\NormalTok{    \}) }
\NormalTok{\}}
\end{Highlighting}
\end{Shaded}

\section{\texorpdfstring{Task: finalize the \texttt{sir\_ode} code above}{Task: finalize the sir\_ode code above}}\label{task-finalize-the-sir_ode-code-above}

Please try to finalize the \texttt{sir\_ode} code above taking care that the function should return a list of gradients, i.e., \(\frac{\mathrm{d}S}{\mathrm{d}t},\frac{\mathrm{d}I}{\mathrm{d}t},\frac{\mathrm{d}R}{\mathrm{d}t}\)

\section{Interfacing the model with the data}\label{interfacing-the-model-with-the-data}

The most important part of this tutorial is to
make a \emph{goal function} which attaches a value to
how well the model fits the data.

We assume
that the number of deaths from the flu data
set above is equal
to the number of infected individuals produced
by the ODE model

To this end, the goal
function should receive the following: (i) the current guess of the parameters \(\beta\) and \(\gamma\)
(for which it will calculate some measure of goodness-of-fit, such as the sum of squares, SS)
(ii) some information about initial densities
and (iii) the actual data of the course of the infection.
Let's do this:

\begin{Shaded}
\begin{Highlighting}[]
\FunctionTok{library}\NormalTok{(}\StringTok{"deSolve"}\NormalTok{)}
\NormalTok{goal.function }\OtherTok{\textless{}{-}} \ControlFlowTok{function}\NormalTok{(parameters, initial\_densities, data)}
\NormalTok{\{}
  \CommentTok{\# first obtain the time points from the dataset}
\NormalTok{  times\_from\_data }\OtherTok{\textless{}{-}}\NormalTok{ data[,}\StringTok{"week"}\NormalTok{]}
  
  \CommentTok{\# solve the SIR ODE over time. }
  \CommentTok{\# store the result as a data.frame}
\NormalTok{  the.ode.data }\OtherTok{\textless{}{-}} \FunctionTok{as.data.frame}\NormalTok{(}\FunctionTok{ode}\NormalTok{(}
    \AttributeTok{y =}\NormalTok{ initial\_densities,}
    \AttributeTok{times =}\NormalTok{ times\_from\_data,}
    \AttributeTok{func =}\NormalTok{ sir\_ode, }\AttributeTok{parms =}\NormalTok{ parameters))}
     
  \CommentTok{\# now extract the numbers of infecteds from }
  \CommentTok{\# the resulting ode data and compare to }
  \CommentTok{\# the ODE}
  \CommentTok{\# and calculate the sum of squares between}
  \CommentTok{\# the data and the ODE}
\NormalTok{\}}
\end{Highlighting}
\end{Shaded}

\section{Task: try to finish the function code above}\label{task-try-to-finish-the-function-code-above}

Try to finalize the final bit of the \texttt{goal.function} code above, by
calculating the sum of squares between
the number of deaths in the \texttt{data} and the density of
infecteds resulting from the ODE, contained in \texttt{the.ode.data}. \emph{Hint} use R's \texttt{sum()}
function to sum over the squared difference between the
data and the ODE.

\section{Task: test driving our function}\label{task-test-driving-our-function}

We can then test-drive our \texttt{goal.function()}
by simply running it with a bunch of arguments
that we made up and see whether there are any
errors.

\begin{Shaded}
\begin{Highlighting}[]
\CommentTok{\# lucky guess of beta and gamma resulting in}
\CommentTok{\# an R0 of 1.25}
\CommentTok{\# we need to provide this as a name{-}value vector}
\NormalTok{pars }\OtherTok{\textless{}{-}} \FunctionTok{c}\NormalTok{(}\AttributeTok{beta =} \DecValTok{1}\NormalTok{, }\AttributeTok{gamma =} \FloatTok{0.8}\NormalTok{)}
\CommentTok{\# lets assume a large number of susceptibles}
\CommentTok{\# but we could potentially vary this later}
\NormalTok{init\_dens }\OtherTok{\textless{}{-}} \FunctionTok{c}\NormalTok{(}\AttributeTok{S=}\FloatTok{1e05}\NormalTok{,}\AttributeTok{I=}\DecValTok{1}\NormalTok{,}\AttributeTok{R=}\DecValTok{0}\NormalTok{)}

\FunctionTok{goal.function}\NormalTok{(}\AttributeTok{parameters =}\NormalTok{ parameters }
\NormalTok{              ,}\AttributeTok{initial\_densities =}\NormalTok{ initial\_densities}
\NormalTok{              ,}\AttributeTok{data =}\NormalTok{ flu}
\NormalTok{              )}
\end{Highlighting}
\end{Shaded}

Now that we have our goal function in hand we
have to feed the goal function to \texttt{optim()}
which helps us find the lowest sum of squares

\begin{Shaded}
\begin{Highlighting}[]
\NormalTok{optim\_result }\OtherTok{\textless{}{-}} \FunctionTok{optim}\NormalTok{(}
  \AttributeTok{par =}\NormalTok{ pars}
\NormalTok{  ,}\AttributeTok{fn =}\NormalTok{ goal.function}
\NormalTok{  ,}\AttributeTok{initial\_densities =}\NormalTok{ init\_dens}
\NormalTok{)}
\end{Highlighting}
\end{Shaded}

\section{Task:}\label{task}

\section{Task: missing data}\label{task-missing-data}

Let's imagine that local health services have been slow
to pick up on the disease, so that the last four data points
are missing. Hence the data set is now as follows:

\begin{Shaded}
\begin{Highlighting}[]
\NormalTok{flu }\OtherTok{\textless{}{-}} \FunctionTok{data.frame}\NormalTok{(}\AttributeTok{week =} \DecValTok{1}\SpecialCharTok{:}\DecValTok{13}\NormalTok{,}
                  \AttributeTok{deaths =} \FunctionTok{c}\NormalTok{(}\DecValTok{14}\NormalTok{, }\DecValTok{28}\NormalTok{, }\DecValTok{50}\NormalTok{, }\DecValTok{66}\NormalTok{, }\DecValTok{156}\NormalTok{, }\DecValTok{190}\NormalTok{, }\DecValTok{156}\NormalTok{, }\DecValTok{108}\NormalTok{, }\DecValTok{68}\NormalTok{, }\DecValTok{77}\NormalTok{, }\DecValTok{33}\NormalTok{, }\DecValTok{65}\NormalTok{, }\DecValTok{24}\NormalTok{))}
\end{Highlighting}
\end{Shaded}

\chapter{Footnotes and citations}\label{footnotes-and-citations}

\section{Footnotes}\label{footnotes}

Footnotes are put inside the square brackets after a caret \texttt{\^{}{[}{]}}. Like this one \footnote{This is a footnote.}.

\section{Citations}\label{citations}

Reference items in your bibliography file(s) using \texttt{@key}.

For example, we are using the \textbf{bookdown} package \citep{R-bookdown} (check out the last code chunk in index.Rmd to see how this citation key was added) in this sample book, which was built on top of R Markdown and \textbf{knitr} \citep{xie2015} (this citation was added manually in an external file book.bib).
Note that the \texttt{.bib} files need to be listed in the index.Rmd with the YAML \texttt{bibliography} key.

The RStudio Visual Markdown Editor can also make it easier to insert citations: \url{https://rstudio.github.io/visual-markdown-editing/\#/citations}

\chapter{Blocks}\label{blocks}

\section{Equations}\label{equations}

Here is an equation.

\begin{equation} 
  f\left(k\right) = \binom{n}{k} p^k\left(1-p\right)^{n-k}
  \label{eq:binom}
\end{equation}

You may refer to using \texttt{\textbackslash{}@ref(eq:binom)}, like see Equation \eqref{eq:binom}.

\section{Theorems and proofs}\label{theorems-and-proofs}

Labeled theorems can be referenced in text using \texttt{\textbackslash{}@ref(thm:tri)}, for example, check out this smart theorem \ref{thm:tri}.

\begin{theorem}
\protect\hypertarget{thm:tri}{}\label{thm:tri}For a right triangle, if \(c\) denotes the \emph{length} of the hypotenuse
and \(a\) and \(b\) denote the lengths of the \textbf{other} two sides, we have
\[a^2 + b^2 = c^2\]
\end{theorem}

Read more here \url{https://bookdown.org/yihui/bookdown/markdown-extensions-by-bookdown.html}.

\section{Callout blocks}\label{callout-blocks}

The R Markdown Cookbook provides more help on how to use custom blocks to design your own callouts: \url{https://bookdown.org/yihui/rmarkdown-cookbook/custom-blocks.html}

\chapter{Sharing your book}\label{sharing-your-book}

\section{Publishing}\label{publishing}

HTML books can be published online, see: \url{https://bookdown.org/yihui/bookdown/publishing.html}

\section{404 pages}\label{pages}

By default, users will be directed to a 404 page if they try to access a webpage that cannot be found. If you'd like to customize your 404 page instead of using the default, you may add either a \texttt{\_404.Rmd} or \texttt{\_404.md} file to your project root and use code and/or Markdown syntax.

\section{Metadata for sharing}\label{metadata-for-sharing}

Bookdown HTML books will provide HTML metadata for social sharing on platforms like Twitter, Facebook, and LinkedIn, using information you provide in the \texttt{index.Rmd} YAML. To setup, set the \texttt{url} for your book and the path to your \texttt{cover-image} file. Your book's \texttt{title} and \texttt{description} are also used.

This \texttt{gitbook} uses the same social sharing data across all chapters in your book- all links shared will look the same.

Specify your book's source repository on GitHub using the \texttt{edit} key under the configuration options in the \texttt{\_output.yml} file, which allows users to suggest an edit by linking to a chapter's source file.

Read more about the features of this output format here:

\url{https://pkgs.rstudio.com/bookdown/reference/gitbook.html}

Or use:

\begin{Shaded}
\begin{Highlighting}[]
\NormalTok{?bookdown}\SpecialCharTok{::}\NormalTok{gitbook}
\end{Highlighting}
\end{Shaded}


  \bibliography{book.bib}

\end{document}
